\section{BASIC DEFINITIONS}
The definition of the integral of a function $f: A\to \B{R}$, where $A\subset \B{R}^n$
is a closed rectangle, is so similar to that of the ordinary integral that 
a rapid treatment will be give.

Recall that a partition $P$ of a closed interval $[a,b]$ is a
sequence $t_0, t_1, \cdots, t_k$, where $a=t_0\le t_1 \le \cdots \le t_k=b$.
The partition $P$ divided the interval into $k$ subintervals $[t_{i-1}, t_i]$.
A \textbf{partition} of a rectangle $[a_1, b_1]\times \cdots \times [a_n, b_n]$
is a collection $P=(P_1, \cdots, P_n)$, where each $P_i$ is a partition of the interval 
$[a_i, b_i]$. Suppose, for example, that $P_1=t_1,\cdots, t_k$ is partition of 
$[a-1, b_1]$ and $P_2=s_0, \cdots, s_l$ is a partition of $[a_2, b_2]$. Then 
$P=(P_1, P_2)$ is a divided the closed rectangle $[a_1, b_1]\times [a_2,b_2]$ into 
$k\times l$ subintervals, a typical one being $[t_{i-1}, t_i]\times [s_{j-1}, s_j]$.
In general, if $P_i$ divides $[a_i, b_i]$ into $N_i$ subintervals, then 
$P=(P_1, \cdots, P_n)$ divides $[a_1, b_1]\times \cdots \times [a_n, b_n]$ into 
$N = N_1 \cdots N_n$ subintervals. These subintervals will be called 
\textbf{subrectangles of the partition} $P$. 

Suppose now that $A$ is a rectangle, $f: A\to \F{R}$ is a bounded function, and $P$ is a 
partition of $A$. For each subrectangle $S$ of the partition let
\begin{align*}
    & {m}_S(f) = \inf \{f(x):x\in S\},\\
    & {M}_S(f) = \sup \{f(x):x\in S\}
\end{align*}

and let $v(S)$ be the volume of $S$ (the \textbf{volume} of a rectangle 
$[a_1, b_1]\times \cdots\times [a_n, b_n]$, and also of $(a_1, b_1)\times \cdots\times (a_n, b_n)$,
is defined as $(b_1-a_1)\cdots (b_n-a_n)$.) The \textbf{lower} and \textbf{upper sums} of $f$ 
for $P$ are defined by
\begin{align*}
    L(f, P) 
    = \sum_{S }^{}{m_S(f)\cdot v(S)} \text{ and } 
    U(f, P) = \sum_{S}^{}{M_S(f)\cdot v(S)}
\end{align*}

Clearly $L(f, P)\le U(f, P)$, and an even stronger assertion (\ref{corollary3-2})
is true.

\begin{lemma}
    Suppose the partition $P'$ refines $P$ (that is, each subrectangle of $P'$
    is contained in a subrectangle of $P$). Then 
    \begin{align*}
        L(f, P) \le L(f, P')
    \end{align*}
    \label{lemma3-1}
\end{lemma}

\begin{proof}
    Each subrectangle $S$ of $P$ is divided into several subrectangles $S_1, \cdots, S_\alpha$ of 
    $P'$, so $v(S) = v(S_1) + \cdots +v(S_\alpha)$. Now $m_S(f)\le m_{S_i}(f)$, since the values 
    $f(x)$ for $s\in S$ included all values $f(x)$ for $x\in S_i$ (and possibly smaller ones).
    Thus 
    \begin{align*}
        m_s(f)\cdot v(S) 
        & = m_S(f)\cdot v(S_1) + \cdots + m_S(f)\cdot v(S_\alpha)\\
        & \le m_{S_1}(f)\cdot v(S_1) + \cdots + m_{S_\alpha}(f)\cdot v(S_\alpha)
    \end{align*}

    The sum, for all $S$, of the terms on the left side is $L(f, P)$, while 
    the sum of the right side is $L(f, P')$. Hence $L(f, P)\le L(f, P')$. The 
    proof of upper sums is similar. 
\end{proof}


\begin{corollary}
    If $P$ and $P'$ are any two partitions, then $L(f,P') \le U(f,P)$.
    \label{corollary3-2}
\end{corollary}

\begin{proof}
    Let $P''$ be a partition which refines both $P$ and $P'$.
    (For example, let $P'' = (P_1^{''}, \cdots, P_n^{''})$, where $P'$ is a 
    partition of $[a_i, b_i]$ which refines $P_i$ and $P_i'$). Then 
    \begin{align*}
        L(f, P') \le L(f, P'') \le U(f, P'') \le U(f, P)
    \end{align*}
\end{proof}

It follows from Corollary 3-2 that the least upper bound of
all lower sums for $f$ is less than or equal to the greatest lower
bound of all upper sums for $f$. A function $f:A \to \B{R}$ is called \textbf{integrable} 
on the rectangle $A$ if $f$ is bounded and $\sup\{L(f,P)\} = \inf \{U(f, P)\}$. 
This common number is then denoted $\int_{A }^{}{f}$, and called the \textbf{integral} of 
$f$ over $A$. Often, the notation $\int_{A  }^{}{f(x^1, \cdots, x^n)  \;\mathrm{d}x^1\cdots \dd x^n}$ 
is used. If $f:[a, b]\to \B{R}$, where $a\le b$, then $\int_{a  }^{b }{f} = \int_{[a, b ]}^{}{f}$.
A simple but useful criterion for integrability is provided by 
\begin{theorem}
    A bounded function $f:A\to \B{R}$ is integrable if and only if for every $\varepsilon>0$ 
    there is a partition $P$ of $A$ such that $U(f, P) - L(f, P)< \varepsilon$.
\end{theorem}

\begin{proof}
    If this condition holds, it is clear that $\sup\{L(f,P)\} = \inf \{U(f, P)\}$ and 
    $f$ is integrable. On the other hand, if $f$ is integrable, so that $\sup\{L(f,P)\} = \inf \{U(f, P)\}$,
    then for any $\varepsilon>0$ there are partitions $P$ and $P'$ with $U(f, P)-L(f,P)<\varepsilon$. If 
    $P''$ refines both $P$ and $P'$, it follows from Lemma \ref{lemma3-1} that $U(f, P'')-L(f, P'')
    \le U(f, P)-L(f, P)<\varepsilon$
\end{proof}

In the following sections we will characterize the integrable
functions and discover a method of computing integrals.
For the present we consider two functions, one integrable and one not.

1. Let $f:A\to \B{R}$ be a constant function, $f(x)=c$. Then for any partition
$P$ and subrectangl $S$ we have $m_S(f) = M_S(f) = c$, so that $L(f, P) = U(f, P)
= \sum_{S}^{}{c\cdot v(S)} = c\cdot v(A)$. Hence $\int_{A  }^{}{f} = c\cdot v(A)$.

2. Let $f:[0, 1]\times [0, 1]\to \B{R}$ be defined by 
\begin{align*}
    f(x, y) =
    \left\{\begin{aligned}
        & 0, \text{ if $x$ is rational} \\
        & 1, \text{ if $x$ is irrational} 
    \end{aligned}\right.
\end{align*}

If $P$ is a partition, then every subrectangle $S$ will be contain points
$(x, y)$ with $x$ rational, and also points $(x, y)$ with $x$ irrational.
Hence 
\begin{align*}
    L(f, P) = \sum_{S  }^{}{0\cdot v(S)} = 0
\end{align*}

and 
\begin{align*}
    U(f, P) = \sum_{S  }^{}{1\cdot v(S)} = v([0, 1]\times [0, 1]) = 1 
\end{align*}

Therefore $f$ is not integrable. 

\begin{problems}
    \problem{
        Let $f:[0,1]\times [0,1]\to \B{R}$ be defined by
        \begin{align*}
            f(x, y) = 
            \left\{\begin{aligned}
                & 0 && \text{ if } 0\le x< \frac12 \\
                & 1 && \text{ if } \frac12\le x\le 1
            \end{aligned}\right.
        \end{align*}

        Show that $f$ is integrable and $\int_{[0,1]\times [0,1]}^{}{f} = \frac12$.
    }
    \problem{Let $f:A\to \B{R}$ be integrable and let $g=f$ except at finitely many points.
    Show that $g$ is integrable and $\int_{A }^{}{f} = \int_A g$.}

    \problem{Let $f,g:A\to \B{R}$ be integrable. 
        \begin{enumerate}[label={\upshape(\alph*)}]
            \item For any partition $P$ of $A$ and subrectangle $S$, show that 
                \begin{align*}
                    m_S(f) + m_S(g) \le m_S(f+g) \text{ and } M_S(f) + M_S(g) \ge M_S(f+g).
                \end{align*}
                and therefore 
                \begin{align*}
                    L(f, P) + L(g, P) \le L(f+g, P) \text{ and } U(f+g, P) \le U(f, P) + U(g, P)
                \end{align*}
            \item Show that $f+g$ is integrable and $\int_{A }^{}{(f+g)} = \int_{A }^{}{f} + \int_{A }^{}{g}$.
            \item For any constant $c$, show that $\int_A cf = c\int_A f$.
        \end{enumerate}
    }
    \problem{
        Let $f: A \to \B{R}$ and let $P$ be a partition of $A$. Show that $f$ is 
        integrable if and only if for each subrectangle $S$ the function $f|S$, which
        consists of $f$ restricted to $S$, is integrable, and that in this case
        $\int_A f = \sum_S\int_S f|S$.
    }
    \problem{
        Let $f,g:A\to \B{R}$ be integrable and suppose $f\le g$. Show that 
        $\int_A f\le \int_A g$. 
    }
    \problem{
        If $f:A\to \B{R}$ is integrable, show that $|f|$ is integrable and 
        $|\int_A f|\le \int_A |f|$.
    }
    \problem{
        Let $f:[0,1]\times [0,1]\to \B{R}$ be defined by 
        \begin{align*}
            f(x,y) = 
            \left\{\begin{aligned}
                & 0 && x \text{ irrational },\\
                & 0 && x \text{ rational and } y \text{ irrational }\\
                & 1/q && x \text{ rational }, y=p/q \text{ in lowest terms }.
            \end{aligned}\right.
        \end{align*}
        Show that $f$ is integrable and $\int_{[0,1]\times [0,1]}^{}{f} = 0$.
    }
\end{problems}
