\documentclass{article}
\PassOptionsToPackage{quiet}{fontspec}
\usepackage{ctex}
\usepackage{xcolor}
\newcommand{\red}[1]{\textcolor{red}{#1}}
\usepackage{hyperref}
\hypersetup{
    colorlinks=true,
    urlcolor=red,
    linkcolor=blue,
}


\begin{document}
\thispagestyle{empty}
本书籍仅作个人使用,\red{切勿用于商业用途};

我排版此书籍的目的很大程度上是为了体验一次\LaTeX{}的排版工作流, 同时也可能是为了给
网络上多增加一份\LaTeX{}的学习资料罢了. 排版是一个极为小众的领域,它并不简单.


本书籍排版所用工具如下:
\begin{itemize}
    \item \red{Simple\TeX{}}: 免费公式识别软件,官网地址:\verb|https://simpletex.cn/|
    \item \red{Inkscape}: 开源矢量图绘制软件,官网地址:\verb|https://inkscape.org/|
    \item \red{Asymptote}: 开源矢量图绘制软件,官网地址:\verb|https://asymptote.sourceforge.io/|
\end{itemize}

排版过程中,主要参考网站:
\begin{itemize}
    \item \red{\TeX{} Stack Exchange}: \verb|https://tex.stackexchange.com/|
    \item \red{Stack Overflow}: \verb|https://stackoverflow.com/|
\end{itemize}

最后再回到这个书籍本身,本文件其实就是按照原版书籍 \red{Calculus on Manifolds -- Michael Spivak},
关于本书籍的更多详细信息请参见:\href{https://en.wikipedia.org/wiki/Calculus\_on\_Manifolds\_(book)}{Wiki}.

本此重排进行了如下的改动:
\begin{itemize}
    \item 把原本书籍中所有图片进行重绘
    \item 部分的符号进行了改写
    \item 以及一些排版上的优化
\end{itemize}

有错是在所难免的,如果各位在参考本资料的同时发现有错误,欢迎指正; 
本人的邮箱地址为: \verb|3552487728@qq.com|

\vspace*{2em}
\noindent\rule{1\linewidth}{3pt}\par
本书内部有部分错误,比如符号不规范,逻辑错误或者是不严密,详情请参见刘思齐老师:
\href{https://www.bilibili.com/video/BV1xp4y1e7Nh/?share\_source=copy\_web\&vd\_source=a9d745bfbf443d7fec13113cdd89ee46}{视频}.

当然,如果有人愿意加入此项目中来,一起改进书中的错误,十分欢迎.(当然,本书籍的 makeindex 工作还没有完成 ...)

\end{document}