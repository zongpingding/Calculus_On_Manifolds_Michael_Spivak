% 生成最终的图象时把第一个文档类取消注释即可
\documentclass[10pt,varwidth]{standalone}
% \documentclass[12pt]{article}
% 1.必须添加varwidth选项,不然就会报错
\PassOptionsToPackage{quiet}{fontspec}
\usepackage{ctex}
% 必须要保证绘图的纸张足够的大
\usepackage[a4paper, left=2.5cm, right=2.5cm, top=2.5cm, bottom=2.5cm]{geometry}
\usepackage{xifthen}
\usepackage{xfp}
\usepackage{xcolor}
\usepackage{pgfplots}
\usepackage{pgfplotstable}
\pgfplotsset{compat=1.16}
% 2.引用的tikz库
\usetikzlibrary {matrix, chains, trees, decorations}
\usetikzlibrary {arrows.meta, automata,positioning}
\usetikzlibrary {decorations.pathmorphing, calc}
\usetikzlibrary {calligraphy}
\usetikzlibrary {backgrounds, mindmap,shadows}
\usetikzlibrary {patterns, quotes, 3d, shadows}
\usetikzlibrary {graphs, fadings, scopes}
\usetikzlibrary {arrows, shapes.geometric}
\usepgflibrary {shadings}

\tikzset{
    >={Latex}
}


\begin{document}
\begin{tikzpicture}
    % axis
    \draw[->] (-5, 0) -- (6, 0)    node [below] {$x$};
    \draw[->] (0, 0)  -- (0, 5)    node [left]  {$z$};
    \draw[->] (-135:4) -- (0, 0) -- (45:7.06) node [above] {$y$};
    % eclipse
    \draw (2,0) arc(0:81:2cm and 1cm);
    \draw (2,0) arc(0:-275:2cm and 1cm);
    % circle
    % \draw (2, 0) circle (2cm);
    \draw (4, 0) arc(0:-149:2cm);
    \draw (4, 0) arc(0:208:2cm);
    % fig num
    \node at (0, -2.5) [below] {(b)};
\end{tikzpicture}    
\end{document}