\section{BASIC DEFINITIONS}
Recall that a function $f:\F{R} \to \F{R}$ is differentiable\index{Differentiable function} at $a\in \F{R}$
if there is a number $f'(a)$ such that 
\begin{align}
    \lim_{h\to 0 }{\frac{f(a+h)-f(a)}{h}} = f'(a)
    \label{eq:differentiation definitins i}
\end{align} 

This equation certainly makes no sense in the general case of a
function $f:\F{R}^n \to \F{R}^m$, but can be reformulated in a way that
does. If $\lambda: \F{R} \to \F{R}$ is the linear transformation defined by
$\lambda(h) = f'(a)\cdot h$, then \eqref{eq:differentiation definitins i} 
is equivalent to 
\begin{align}
    \lim_{h\to 0 }{\frac{f(a+h)-f(a)-\lambda(h)}{h}} = 0
    \label{eq:differentiation definitins ii}
\end{align} 

\eqref{eq:differentiation definitins ii} is often interpreted as saying that 
$\lambda + f(a)$ is a good approximatin to $f$ at $a$ (see Problem 2-9).
Henceforth we focus our attention on the linear transformation $\lambda$ and
reformulate the definition of differentiability as follows.

A function $f:\F{R} \to \F{R}$ is differentiable\index{Differentiable function} at $a\in \F{R}$
if there is a linear transformation $\lambda:\F{R}\to \F{R}$ such that 
\begin{align}
    \lim_{h\to 0 }{\frac{f(a+h)-f(a)-\lambda(h)}{h}} = 0
    \label{eq:differentiation definitins iii}
\end{align} 

In this form the definition has a simple generalization to
higher dimensions:

A function $f:\F{R}^n \to \F{R}^m$ is differentiable at $a\in \F{R}$
if there is a linear transformation $\lambda:\F{R}^n\to \F{R}^m$ such that
\begin{align}
    \lim_{h\to 0 }{\frac{|f(a+h)-f(a)-\lambda(h)|}{|h|}} = 0
    \label{eq:differentiation definitins iiii}
\end{align} 

Note that $h$ is a point of $\F{R}^n$ and $f(a+h)-f(a)-\lambda(h)$ a
point of $\F{R}^m$, so the norm signs are essential. The linear 
transformation $\lambda$ is denoted $\R{D}f(a)$ and called the \textbf{derivative}\index{Derivative} of 
$f$ at $a$. The justification for the phrase "the linear transformation $\lambda$" is.

\begin{theorem}
    If $f:\F{R}^n\to \F{R}^m$ is differentiable at $a\in \F{R}^n$ there is a unique linear 
    transformation $\lambda: \F{R}^n\to \F{R}^m$ such that
    \begin{align*}
        \lim_{h\to 0 }{\frac{|f(a+h)-f(a)-\lambda(h)|}{|h|}} = 0
    \end{align*}
\end{theorem}

\begin{proof}
    Suppose $\mu:\F{R}^n\to \F{R}^m$ satisfies 
    \begin{align*}
        \lim_{h\to 0 }{\frac{|f(a+h)-f(a)-\lambda(h)|}{|h|}} = 0
    \end{align*}

    If $d(h) = f(a+h) - f(a)$, then
    \begin{align*}
        \lim_{h\to 0 }{|\lambda(h)-\mu(h)|}{|h|}
        & = \lim_{h\to 0}{\frac{|\lambda(h) - d(h) + d(h) - \mu(h)|}{|h|}} \\
        & \le \lim_{h\to 0}{\frac{|\lambda(h) - d(h)}{|h|}} + \lim_{h\to 0}{\frac{|d(h) - \mu(h)|}{|h|}} \\
        & = 0
    \end{align*}

    If $x\in \F{R}^n$, then $tx\to 0$ as $t\to 0$, Hence for $x\neq 0$ we have 
    \begin{align*}
        0 = \lim_{h\to 0}{\frac{|\lambda(tx) - \mu(tx)|}{|th|}} = \frac{|\lambda(x) - \mu(x)|}{|x|}
    \end{align*}

    Therefore $\lambda(x) = \mu(x)$.
\end{proof}

We shall later discover a simple way of finding $\R{D}f(a)$. For
the moment let us consider the function $f:\F{R}^2 \to \F{R}$defined by
$f(x,y) = sin x$. Then $\R{D}f(a,b) = \lambda$ satisfies $\lambda(x,y) = (\cos a)\cdot x$.
To prove this, note that
\begin{align*}
    \lim_{(h, k)\to 0}{\frac{|f(a+h, b+k) - f(a, b) - \lambda(h, k)|}{|(h, k)|}}\\
        = \lim_{(h, k)\to 0}{\frac{|\sin(a+h) - \sin a - (\cos a)\cdot h|}{|(h, k)|}}
\end{align*}

Since $\sin'(a) = \cos a$, we have 
\begin{align*}
    \lim_{h\to 0}{\frac{|\sin(a+h) - \sin a - (\cos a)\cdot h|}{|h|}} = 0
\end{align*}

Since $|(h, k)| \ge |h|$, it is also true that 
\begin{align*}
    \lim_{h\to 0}{\frac{|\sin(a+h) - \sin a - (\cos a)\cdot h|}{|(h, k)|}} = 0
\end{align*}

It is often convenient to consider the matrix of $\R{D}f(a): \F{R}^n\to \F{R}^m$ 
with respect to the usual bases of $\F{R}^n$ and $\F{R}^m$. This $m\times n$ 
matrix is called the \Index{Jacobian matrix}\index{Matrix!jacobian} of $f$ at $a$, and denoted $f'(a)$.
If $f(x, y) = \sin x$, then $f'(a, b) = (\cos a, 0)$. If $f:\F{R}\to \F{R}$, then 
$f'(a)$ is a $1\times 1$ mastrix whose single entry is the number which is denoted
$f'(a)$ in the elementary calculus.

The definition of $\R{D}f(a)$ could be made iff were defined only
in some open set containing $a$. Considering only functions
defined on $\F{R}^n$ streamlines the statement of theorems and
produces no real loss of generality. It is convenient to define
a function $f:\F{R}^n\to \F{R}^m$ to be differentiable on $A$ if $f$ 
is differentiable at $a$ for each $a\in A$. If $f:A \to \F{R}^m$, 
then $f$ is called differentiable if $f$ can be extended to a 
differentiable function on some open set containing $A$.

\begin{problems}
    \problem{Prove that if $f:\F{R}^n\to \F{R}^m$ is differentiable at
        $a \in \F{R}^m$, then it is continuous at $a$. \textit{Hint:} Use Problem 1-10.}
    \problem{A function $f:\F{R}^2 \to \F{R}$ is \textbf{independent of the second variable}\index{Variable!independent of the second}
        if for each $x\in \F{R}$ we have $f(x, y_1) = f(x, y_2)$ for all $y_1, y_2\in \F{R}$
        Show that $f$ is independent of the second variable if and only if there is a
        function $g:\F{R}\to \F{R}$ such that $f(x,y) = g(x)$. What is $f'(a,b)$ in
        terms of $g'$?}
    \problem{Define a function $f:\F{R}^2 \to \F{R}$ is independent of the first\index{Variable!independent of the first} 
        variable and find $f'(a,b)$ for such $f$. Which functions are independent of
        the first variable and also of the second variable?}
    \problem{Let $g$ be a continuous real-valued function on the unit circle $\{x\in \F{R}^2:|x|=1\}$ 
        such that $g(0, 1) = g(1, 0) = 0$ and $g(-x) = -g(x)$. Define $f:\F{R}^2\to \F{R}$ by 
        \begin{align*}
            f(x) = 
            \left\{\begin{aligned}
                % & |x| \cdot g\left(\begin{matrix}x\\|x|\end{matrix}\right) && x\neq 0 \\
                % & |x| \cdot g\left(\substack{x\\|x|}\right) && x\neq 0 \\
                & |x| \cdot g\left(\frac{x}{|x|}\right) && x\neq 0 \\
                & 0 && x = 0
            \end{aligned}\right.
        \end{align*}

        \begin{enumerate}[label={\upshape(\alph*)}]
            \item If $x\in \F{R}^2$ and $h:\F{R}\to \F{R}$ is defined by $h(t) = f(tx)$, show that $h$ is differentiable.
            \item Show that $f$ is not differentiable at $(0, 0)$ unless $g=0$ \textit{Hint:} First show that $\R{D}f(0, 0)$ 
                would have to be $0$ by considering $(h, k)$ with $k=0$ and then with $h=0$. 
        \end{enumerate}
        }
    \problem{Let $f:\F{R}^2 \to \F{R}$ be defined by 
            \begin{align*}
                f(x, y) = 
                \left\{\begin{aligned}
                    & \frac{x|y|}{\sqrt{x^2 + y^2}} && (x, y) \neq 0 \\
                    & 0 && (x, y) = 0.
                \end{aligned}\right.
            \end{align*}

            Show that $f$ is a function of the kind considered in Problem 2-4, so that 
            $f$ is not differentiable at $(0, 0)$.
        }
    \problem{Let $f:\F{R}^2\to \F{R}$ be defined by $f(x, y) = \sqrt{|xy|}$. Show that $f$ is not differentiable at (0, 0).}
    \problem{Let $f:\F{R}^2\to \F{R}$ be a function such that $|f(x)|\le |x|^2$. Show that $f$ is differentiable at 0.}
    \problem{Let $f:\F{R}^2\to \F{R}$. Prove that $f$ is differentiable at $a\in \F{R}$ if and only if $f^1$ and $f^2$ 
        are, and that in  this case 
            \begin{align*}
                f'(a) = \left\{\begin{matrix}
                    \left(f^1\right)'(a) \\ 
                    \left(f^2\right)'(a)
                \end{matrix}\right.
            \end{align*}
        }
    \problem{Two functions $f,g: \F{R}\to \F{R}$ are \Index{Equal up to $n$th order} at $a$ if and only if 
            \begin{align*}
                \lim_{h\to 0}{\frac{f(a+h) - g(a+h)}{h^n}} = 0
            \end{align*}

            \begin{enumerate}[label={\upshape(\alph*)}]
                \item Show that $f$ is differentiable at $a$ if and only if there is a function $g$ of the 
                    form $g(x) = a_0 + a_1(x-a)$ such that $f$ and $g$ are equal up to first order at $a$.
                \item If $f'(a), \cdots, f^{(n)}(a)$ exits, show that $f$ and the function $g$ defined by 
                    \begin{align*}
                        g(x) = \sum_{i=0}^{n}{\frac{f^{(i)}}{i!}}(x-1)^i 
                    \end{align*}\relax 
                    are equal up to the $n$th order at $a$. \textit{Hint:} The limit
                    \begin{align*}
                        \lim_{x\to a}{\frac{\displaystyle f(x) - \sum_{i=0}^{n-1}{\frac{f^{i}(a)}{i!}(x-a)^i}}{(x-a)^n}}
                    \end{align*}\relax 
                    may be evaluated by L'Hospital rule.
            \end{enumerate}
        }
\end{problems}