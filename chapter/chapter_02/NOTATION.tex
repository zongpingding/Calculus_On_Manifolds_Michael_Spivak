\clearpage
\section[\textsc{notation}]{NOTATION}\index{Notation}
This section is a brief and not entirely unprejudiced discussion
of classical notation connected with partial derivatives.

The partial derivative $D_1f(x,y,z)$ is denoted, among devotees
of classical notation, by
\begin{align*}
    \frac{\partial f(x,y,z)}{\partial x}
    \quad\mathrm{or}\quad
    \frac{\partial f}{\partial x}
    \quad\mathrm{or}\quad
    \frac{\partial f}{\partial x}(x,y,z)
    \quad\mathrm{or}\quad
    \frac\partial{\partial x}f(x,y,z)
\end{align*}

or any other convenient similar symbol. This notation forces
one to write
\begin{align*}
    \frac{\partial f }{\partial u}(u,v,w)
\end{align*}

for $D_1f(u, v, w)$, although the symbol 
\begin{align*}
    \left.\frac{\partial f(x,y,z)}{\partial x}\right|_{(x,y,z)=(u,v,w)}
    \quad\mathrm{or}\quad
    \frac{\partial f(x,y,z)}{\partial x}\left(u,v,w\right)
\end{align*}

or something similar may be used (and must be used for an
expression like $\R D_1f(7,3,2)$. Similar notation is used for $\R D_2f$
and $\R D_3f$. Higher-order derivatives are denoted by symbols
like
\begin{align*}
    \R D_2\R D_1f(x,y,z)=\frac{\partial^2f(x,y,z)}{\partial y\partial x}.
\end{align*}

When $f\colon{}\F{R}\to \F{R}$, the symbol $\partial$ automatically reverts to $\dd$; thus
\begin{align*}
    \frac{\dd\sin x}{\dd x}
    \quad\mathrm{not}\quad
    \frac{\partial\sin x}{\partial x}.
\end{align*}

The mere statement of Theorem 2-2 in classical notation
requires the introduction of irrelevant letters. The usual
evaluation for $\R D_1(f\circ(g, h))$ runs as follows:

If $f(u,v)$ is a function and $u=g(x, y)$ and $v=h(x, y)$, then
\begin{align*}
    \frac{\partial f(g(x,y),h(x,y))}{\partial x}
    =
    \frac{\partial f(u,v)}{\partial u}\frac{\partial u}{\partial x}+\frac{\partial f(u,v)}{\partial v}\frac{\partial v}{\partial x}.
\end{align*}

[The symbol $\partial u/\partial x$ means $\partial/\partial x$ $g(x,y)$ 
and $\partial/\partial uf(u,v)$ means $\R D_1f(u,v)=\R D_1f(g(x,y),h(x,y))$.]

This equation is often written simply
\begin{align*}
    \frac{\partial f}{\partial x}
    =\frac{\partial f}{\partial u}\frac{\partial u}{\partial x}
        +\frac{\partial f}{\partial v}\frac{\partial v}{\partial x}\cdotp 
\end{align*}

Note that $f$ means something different on the two sides of the
equation!

The notation $\dd f/\dd x$, always a little too tempting, has inspired
many (usually meaningless) definitions of $\dd x$ and $\dd f$ separately,
the sole purpose of which is to make the equation
\begin{align*}
    \dd f=\frac{\dd f}{\dd x}\cdot \dd x
\end{align*}

work out. If $f\colon{}\F{R}^2\to \F{R}$ then $\dd f$ is defined, classically, as 
\begin{align*}
    \dd f=\frac{\partial f}{\partial x}\dd x+\frac{\partial f}{\partial y}\dd y
\end{align*}

(whatever $\dd x$ and $\dd y$ mean).

Chapter 4 contains rigorous definitions which enable us to
prove the above equations as theorems. It is a touchy
question whether or not these modern definitions represent a
real improvement over classical formalism; this the reader
must decide for himself.