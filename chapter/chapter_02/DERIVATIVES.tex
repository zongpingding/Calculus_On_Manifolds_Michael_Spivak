\clearpage
\section{DERIVATIVES}
The reader who has compared Problems 2-10 and 2-17 has
probably already guessed the following.

\begin{theorem}
    If $f:\F{R}^n\to \F{R}^m$ is differentiable at $a$, then $\R{D}_if^i(a)$ exists for
    $1\le i \le m, 1\le j\le n$ and $f'(a)$ is the $m\times n$ matrix $\left(\R{D}_jf^i(a)\right)$.
\end{theorem}

\begin{proof}
    Suppose first that $m=1$, so that $f:\F{R}^n\to \F{R}$. Define $h:\F{R}\to \F{R}^n$ by 
    $h(x) = \left(a^1, \cdots, x, \cdots, a^n\right)$, with $x$ in the $j$th place, Then 
    $\R{D}_jf(a) = (f\circ h)'(a^j)$. Hence by Theorem 2-2,
    \begin{align*}
        (f\circ h)'(a^j) 
        = f'(a)\cdot h'(a^j) 
        = f'(a) \cdot \left(\begin{matrix}
            0 \\ \vdots \\ 1 \\ \vdots \\ 0
        \end{matrix}\right)
        \leftarrow j\text{th place}
    \end{align*} 
    Since $(f\circ h)'(a^j)$ has the single entry $\R{D}_jf(a)$, this shows that $\R{D}_jf(a)$ exists and is the 
    $j$th entry of the $1\times n$ matrix $f'(a)$.

    The theorem now follows for arbitary $m$ since, by Theorem 2-3, each $f^i$ is differentiable and the 
    $i$th row of $f'(a)$ is $(f^j)'(a)$.
\end{proof}

There are several examples in the problems to show that the
converse of Theorem 2-7 is false. It is true, however, if one
hypothesis is added.

\begin{theorem}
    If $f:\F{R}^n\to \F{R}^n$, then $\R{D}f(a)$ exists if all $\R{D}_jf(a)$ exists in an open set containing 
    $a$ and if each function $\R{D}_jf^i$ is continuous at $a$.
\end{theorem}

\begin{proof}
    As in the proof of Theorem 2-7, it suffices to consider the case $m = 1$, so that 
    $f:\F{R}^n\to \F{R}$. Then 
    \begin{align*}
        f(a+h) &  - f(a) = f(a^1+h^1, a^2, \cdots, a^n) - f(a^1, \cdots, a^n) \\
        & + f(a^1+h^1, a^2+h^2, a^3, \cdots, a^n) - f(a^1+h^1, a^2, \cdots, a^n) \\
        & + \cdots \\
        & + f(a^1+h^1, \cdots, a^n+h^n) - f(a^1+h^1, \cdots, a^{n-1}+h^{n-1}, a^n)
    \end{align*}
    Recall that $\R{D}_jf$ is the derivative of the function $g$ defined by $g(x) = f(x, a^2, \cdots, a^n)$. Applying 
    the mean-value theorem to $g$ obtain 
    \begin{align*}
        f(a^1+h^1, a^2, \cdots, a^n) - f(a^1, \cdots, a^n)
        = h^1\cdot \R{D}_1f(b_1, a^2, \cdots, a^n)
    \end{align*}
    for some $b_1$ between $a^1$ and $a^1+h^1$. Similarly, the $i$th term in the sum equals
    \begin{align*}
        h^i\cdot \R{D}_if(a^1+h^1, \cdots, a^{i-1}+h^{i-1}, b_i, \cdots, a^n)
        = h^i\R{D}_if(c_i)
    \end{align*}
    for some $c_i$. Then 
    \begin{align*}
        &\lim_{h\to 0}{\frac{\left|f(a+h) - f(a) - \sum_{i=1}^{n }{\R{D}_if(a)\cdot h^i}\right|}{|h|}} \\
        = & \lim_{h\to 0}{\frac{\left|\sum_{i=1}^{n }{\left[\R{D}_if(c_i) - \R{D}_if(a)\right]\cdot h^i}\right|}{|h|}} \\
        \le & \lim_{h\to 0}{\sum_{i=1}^{n }{\left|\R{D}_if(c_i) - \R{D}_if(a)\right|\cdot \frac{|h^i|}{|h|}}} \\
        \le & \lim_{h\to 0}{\sum_{i=1}^{n }{\left|\R{D}_if(c_i) - \R{D}_if(a)\right|}} \\
        = & 0
    \end{align*}
    since $\R{D}_if$ is continuous at $a$.
\end{proof}

Although the chain rule was used in the proof of Theorem
2-7, it could easily have been eliminated. With Theorem 2-8 to
provide differentiable functions, and Theorem 2-7 to provide
their derivatives, the chain rule may therefore seem almost
superfluous. However, it has an extremely important corollary 
concerning partial derivatives.

\begin{theorem}
    Let $g_1, \cdots, g_m:\F{R}^n\to \F{R}$ be continuously differentiable at $a$, and 
    let $f:\F{R}^m\to \F{R}$ be differentiable at $\left(g_1(a), \dots, g_m(a)\right)$. 
    Define the function $F:\F{R}^n\to \F{R}$ by $F(x) = f\left(g_1(x), \cdots, g_m(x)\right)$. Then 
    \begin{align*}
        \R{D}_iF(a) = \sum_{j=1}^{m}{\R{D}_jf\left(g_a(a), \cdots, g_m(a)\right)\cdot \R{D}_ig_j(a)}
    \end{align*}
\end{theorem}

\begin{proof}
    The function $F$ is just the composition $f\circ g$, where $g = (g_1,\cdots,g_m)$.
    Since $g_i$ is continuously differentiable at $a$, it follows from Theorem 2-8 
    that $g$ is differentiable at $a$. Hence by Theorem 2-2,
    \begin{align*}
        F'(a) 
        & = f'(g(a)) \cdot g'(a) \\
        & = \left(\R{D}_1f(g(a)), \cdots, \R{D}_mf(g(a))\right)\cdot 
            \left(\begin{matrix}
                \R{D}_1g_1(a) & \cdots & \R{D}_ng_1(a) \\
                \vdots & & \vdots \\
                \R{D}_1g_m(a) & \cdots & \R{D}_ng_m(a)
            \end{matrix}\right)
    \end{align*}
    But $\R{D}_iF(a)$ is the $i$th entry of the left side of this equation, while 
    $\sum_{j=1}^{m }{\R{D}_jf\left(g_1(a), \cdots, g_m(a)\right)}\cdot \R{D}_ig_j(a)$ is the $i$th 
    entry of the right side.
\end{proof}

Theorem 2-9 is often called the chain rule, but is weaker
than Theorem 2-2 since $g$ could be differentiable without $g_i$
being continuously differentiable (see Problem 2-32).
Most computations requiring Theorem 2-9 are fairly straightforward.
A slight subtlety is required for the function $F:\F{R}^2\to \F{R}$
defined by 
\begin{align*}
    F(x, y) = f\left(g(x, y), h(x), k(y)\right)
\end{align*}

where $h, k:\F{R}\to \F{R}$. In order to apply Theorem 2-9 define $\hat{h}, \hat{k}:\F{R}^2\to \F{R}$ by 
\begin{align*}
    \hat{h}(x, y) = h(x) & & \hat{k}(x, y) = k(y).
\end{align*}
Then 
\begin{align*}
    & \R{D}_1\hat{h}(x, y) = h'(x) && \R{D}_2\hat{h}(x, y) = 0 \\
    & \R{D}_1\hat{k}(x, y) = 0 && \R{D}_2\hat{k}(x, y) = k'(y)
\end{align*}

and we can write 
\begin{align*}
    F(x, y) = f\left(g(x, y), \hat{h}(x, y), \hat{k}(x, y)\right)
\end{align*}

Letting $a= \left(g(x, y), h(x), k(y)\right)$, we obtain
\begin{align*}
    & \R{D}_1F(x, y) = \R{D}_1f(a)\cdot \R{D}_1g(x, y) + \R{D}_2f(a)\cdot h'(x)\\
    & \R{D}_2F(x, y) = \R{D}_1f(a)\cdot \R{D}_2g(x, y) + \R{D}_3f(a)\cdot k'(y)
\end{align*}

It should, of course, be unnecessary for you to actully write 
down the function $\hat{h}$ and $\hat{k}$

\begin{problems}
    \problem{Find expressions for the partial derivatives of the following functions:
        \begin{enumerate}[label={\upshape(\alph*)}]
            \item $F(x,y)=f(g(x)k(y),g(x)+h(y))$
            \item $F(x,y,z)=f(g(x+y),h(y+z))$
            \item $F(x,y,z)=f(x^{y},y^{z},z^{x})$
            \item $F(x,y)=f(x,g(x),h(x,y))$
        \end{enumerate}
        }
    \problem{Let $f:\F{R}^n\to \F{R}$. For $x\in \F{R}^n$, the limit 
            \begin{align*}
                \lim_{t\to 0}{\frac{f(a+tx) - f(a)}{t}}
            \end{align*}
            if exists, is denoted $\R{D}_xf(a)$, and called the \textbf{directional derivative} of $f$ at $a$,
            in the direction $x$.
            \begin{enumerate}[label={\upshape(\alph*)}]
                \item Show that $\R{D}_{e_i}f(a) = \R{D}_if(a)$
                \item Show that $\R{D}_{tx}f(a) = t\R{D}_xf(a)$
                \item If $f$ is differentiable at $a$, show that $\R{D}_xf(a) = \R{D}f(a)(x)$ and therefore 
                    $\R{D}_{x+y}f(a) = \R{D}_xf(a) = \R{D}_yf(a)$.
            \end{enumerate}
        }
    \problem{Let $f$ be defined as in the Problem 2-4. Show that $\R{D}_xf(0, 0)$ exists for all $x$
        but if $g\neq 0$, then $\R{D}_{x+y}f(0, 0) = \R{D}_xf(0, 0) + \R{D}_yf(0, 0)$ is not true for all $x$ and $y$.}
    \problem{Let $f:\F{R}^2\to \F{R}$ be defined as in Problem 2-26. Show that $\R{D}_xf(0, 0)$ exists for all 
        $x$ , although $f$ is not even continuous at $(0, 0)$.}
    \problem{
        \begin{enumerate}[label={\upshape(\alph*)}]
            \item Let $f:\F{R}\to \F{R}$ be defined by 
                \begin{align*}
                    f(x) = \left\{\begin{aligned}
                        & x^2\sin \frac{1}{x} && x\neq 0 \\
                        & 0 && x = 0
                    \end{aligned}\right.
                \end{align*}
                Show that $f$ is differentiable at 0 but $f'$ is not continuous at 0.
            \item Let $f:\F{R}^2\to \F{R}$ be defined by 
                \begin{align*}
                    f(x, y) = \left\{\begin{aligned}
                        & (x^2 + y^2)\sin \frac{1}{\sqrt{x^2 + y^2}} && (x, y)\neq 0 \\
                        & 0 && (x, y) = 0
                    \end{aligned}\right.
                \end{align*}
                Show that $f$ is differentiable at 0 but $\R{D}_if$ is not continuous at (0, 0).
        \end{enumerate}
        }
    \problem{Show that the continuity of $\R{D}_1f^i$ at a may be eliminated from the hypothesis of Theorem 2-8.}
    \problem{A function $f:\F{R}^n\to \F{R}$ is \textbf{homogeneous} of degree $m$ if $f(tx) = t^mf(x)$ 
        for all $x$. If $f$ is also differentiable , show that 
        \begin{align*}
            \sum_{i=1}^{n }{x^i\R{D}_if(x)} = mf(x)
        \end{align*}
        \textit{Hint:} If $g(t) = f(tx)$, find $g'(1)$.
        }
    \problem{If $f:\F{R}^n\to \F{R}$ is differentiable and $f(0) = 0$, prove that there exists $g_i:\F{R}^n\to \F{R}$
        such that 
        \begin{align*}
            f(x) = \sum_{i=1}^{n}{x^ig_x(x)}
        \end{align*}
        \textit{Hint:} If $h_x(t) = f(tx)$, then $f(x) = \int_{0}^{1}{h_x'(t) \mathrm{d}t}$.
        }
\end{problems}