\clearpage
\section{STOKES' THEOREM ON MANIFOLDS}
If $\omega$ is a $p$-form on a $k$-dimensional manifold-with-boundary
$M$ and $c$ is a singular $p$-cube in $M$, we define
\begin{align*}
    \int_c \omega = \int_{[0,1]^p} c^*\omega
\end{align*}

precisely as before; integrals over $p$-chains are also defined as
before. In the case $p = k$ it may happen that there is an
open set $W\supset [0,1]^k$ and a coordinate system $f:W\to\F{R}^n$ such
that $c(x) = f(x)$ for $x\in [0,1]^k$; a $k$-cube in $M$ will always be
understood to be of this type. If $M$ is oriented, the singular
$k$-cube $c$ is called \textbf{orientation-preserving}\index{Orientation-preserving} if $f$ is.

\index{Integral!of a form on a manifold|(}
\begin{theorem}
    If $c_1,c_2: [O,1]^k\to M$ are two orientation-preserving singular $k$-cubes in the 
    oriented $k$-dimensional manifold $M$ and $\omega$ is a $k$-form on $M$ such 
    that $\omega= 0$ outside of $c_1([0,1]^k) \cap c_2([0,1]^k)$, then
    \begin{align*}
        \int_{c_1}\omega = \int_{c_2} \omega
    \end{align*}
\end{theorem}

\begin{proof}
    We have 
    \begin{align*}
        \int_{c_1}\omega 
        = \int_{[0,1]^k} c_1^*(\omega) 
        = \int_{[0,1]^k} (c_2^{-1}\circ c_1)^*c_2^*(\omega)
    \end{align*}

    (Here $c_2^{-1}\circ c_1$ is defined only on a subset of $[0,1]^k$ and the second equality depends on the 
    fact that $\omega=0$ outside of $c_1([0,1]^k)\cap c_2([0,1]^k)$.) It thereforw suffices to show that 
    \begin{align*}
        \int_{[0,1]^k}(c_2^{-1}\circ c_1)^*c_2^*(\omega)=\int_{[0,1]^k}c_2^*(\omega)=\int_{c_2}\omega
    \end{align*}

    if $c_2^*(\omega) =d\dd x^1\wedge\cdots\wedge\dd x^k$ and $c_2^{-1}\circ c_1$ is denoted 
    by $g$, then byy Theorem 4-9 we have 
    \begin{align*}
        (c_2^{-1}\circ c_1)^*c_2^*(\omega)
        & = g^*(f\;\dd x^1\wedge\cdots\wedge \dd x^k) \\
        & = (f\circ g)\cdot\det g'\cdot \dd x^1\wedge\cdots\wedge \dd x^k\\
        & = (f\circ g)\cdot\left|\det g'\right|\cdot \dd x^1\wedge\cdots\wedge \dd x^k
    \end{align*}

    since $\det g'(c_2^{-1}\circ c_1)>0$. The result now follows from Theorem 3-13.
\end{proof}

The last equation in this proof should help explain why we
have had to be so careful about orientations.

Let $\omega$ be a $k$-form on an oriented $k$-dimensional manifold $M$.
If there is an orientation-preserving singular $k$-cube $c$ in $M$ such
that $\omega=0$ outside of $c([0, 1]^k)$, we define
\begin{align*}
    \int_M\omega=\int_c\omega
\end{align*}

Theorem 5-4 shows $\int_M \omega$ does not depend on the choice of $c$. Suppose 
now that $\omega$ is an arbitrary $k$-form on $M$. There is an open cover $\C{O}$
of $M$ such that $U\in \C{O}$ there is an orientation-preserving singular $k$-cube 
$c$ with $U\subset c([0,1]^k)$. Let $\Phi$ be a partition of unity for $M$ suboridinate 
to this cover. We define 
\begin{align*}
    \int_M\omega=\sum_{\varphi\in\Phi}\int_M\varphi\cdot\omega 
\end{align*} 

provided the sum converges as described in the discussion pre-
ceding Theorem 3-12 (this is certainly true if $M$ is compact).
An argument similar to that in Theorem 3-12 shows that $\int_M \omega$
does not depend on the cover $\C{O}$ or on $\Phi$.

All our definitions could have been given for a $k$-dimensional manifold-with-boundary $M$ 
with orientation $\mu$ Let $\partial M$ have the induced orientation $\partial\mu$.
Let $c$ be an orientation-preserving $k$-cube in $M$ such that $c_{(k,O)}$ lies in $\partial M$ and 
is the only face which has any interior points in $\partial M$. As the remarks after
the definition of $\partial\mu$ show, $c_{(k.O)}$ is orientation-preserving if $k$ is
even, but not if $k$ is odd. Thus, if $\omega$ is a $(k-1)$-form on $M$ which is 0 outside 
of $c_{([0,1]^k)}$, we have
\begin{align*}
    \int_{c(k,0)}\omega = (-1)^k\int_{\partial M}\omega
\end{align*}

On the other hand, $c_{(k,O)}$ appears with coefficient $(-1)^k$ in $\partial c$.
Therefore
\begin{align*}
    \int_{\partial c}\omega 
    = \int_{(-1)^{k}c_{(k,0)}}\omega
    = (-1)^{k} \int_{c_{(k,0)}}\omega
    = \int_{\partial M}\omega
\end{align*}

Our choice of $\partial \mu$ was made to eliminate any minus signs in this equation, and 
in the following theorem.
\index{Integral!of a form on a manifold|)}

\begin{theorem}[Stokes' Theorem]\index{Stokes' Theorem}
    If $M$ s a compact oriented $k$-dimensional manifold-with-boundary and $\omega$ is a
    $(k-1)$-form on $M$, then
    \begin{align*}
        \int_M\dd\omega = \int_{\partial M}\omega
    \end{align*}
\end{theorem}

(Here $\partial M$ is given the induced orientation.)


\begin{proof}
    Suppose first that there is an orientation-preserving singular $k$-cube in $M-\partial M$ such that 
    $\omega=0$ outside of $c([0,1]^k)$. By Theorem 4-13 and the definition of $\dd\omega$ we have 
    \begin{align*}
        \int_{c}\;\dd\omega
        = \int_{[0,1]^k}c^*(\dd\omega)
        = \int_{[0,1]^k}\;\dd(c^*\omega)
        = \int_{\partial I^k}c^*\omega
        = \int_{\partial c}\omega       
    \end{align*} 

    Then 
    \begin{align*}
        \int_M\;\dd\omega
        = \int_c\;\dd\omega
        = \int_{\partial c}\omega
        = 0        
    \end{align*}

    since $\omega=0$ on $\partial c$. On the other hand, $\int_{\partial M}\omega =0$ since 
    $\omega=0$ on $\partial M$.

    Suppose next that there is an orientation-preserving singular $k$-cube in $M$ such that 
    $c_{(k,0)}$ is the only face in $\partial M$, and $\omega=0$ outside of $c([0,1]^k)$. Then
    \begin{align*}
        \int_{{M}}\;\dd\omega
        = \int_{{c}}\;\dd\omega
        = \int_{{\partial c}}\omega
        = \int_{{\partial M}}\omega        
    \end{align*}

    Now consider the general case. There is an open cover $\C{O}$ of $M$ and a partition of unity
    $\Phi$ for $M$ suboridinate to $\C{O}$ such that for each $\varphi\in\Phi$ the form $\varphi\cdot\omega$
    is of one of the two sorts already considered. We have 
    \begin{align*}
        0 = \dd(1) = \dd\bigg(\sum_{\varphi\in\Phi}\varphi\bigg)
        = \sum_{\varphi\in\Phi}\;\dd\varphi
    \end{align*}

    so that 
    \begin{align*}
        \sum_{\varphi\in\phi}^{}{\int_M \;\dd\varphi\wedge\omega} = 0
    \end{align*}

    Therefore 
    \begin{align*}
        \int_M\;\dd\omega
        & = \sum_{\varphi\in\Phi}\int_M\varphi\cdot \dd\omega
            = \sum_{\varphi\in\Phi}\int_M\;\dd\varphi\wedge\omega+\varphi\cdot \dd\omega \\
        & = \sum_{\varphi\in\Phi}\int_M\;\dd(\varphi\cdot\omega)
            = \sum_{\varphi\in\Phi}\int_M\varphi\cdot\omega  \\
        & = \int_{\partial M}\omega
    \end{align*}
\end{proof}

\begin{problems}
    \problem{
        If $M$ is an $n$-dimensional manifold (or manifold-with-boundary) in $\F{R}^n$, with the 
        usual orientation, show that $\int_M f\;\dd x^1\wedge\cdots\wedge\dd x^n$, as defined in this 
        section, is the same as $\int_M f$, as defined in Chapter 3.
    }
    \problem{
        \begin{enumerate}[label=(\alph*)]
            \item Show that Theorem 5-5 is false if $M$ is not compact.
                Hint: If $M$ is a manifold-with-boundary for which 5-5 holds, then $M-\partial M$
                is also a manifold-with-boundary (with empty boundary).
            \item Show that Theorem 5-5 holds for noncompact $M$ provided
                that $\omega$ vanishes outside of a compact subset of $M$.
        \end{enumerate}
    }
    \problem{
        If $\omega$ is a $(k-1)$-form on a compact $k$-dimensional manifold $M$, prove that
        $\int_M \;\dd\omega$. Give a counterexample if $M$ is not compact.
    }
    \problem{
        An \textbf{absolute $k$-tensor}\index{Tensor!absolute}\Index{Absolute tensor} on $V$ is a function $\eta:V^k\to \F{R}$ of the 
        form $|\omega|$ for $\omega\in\Lambda^k(V)$. An \textbf{absolute $k$-form}\index{Differential form!absolute}\Index{Absolute differential form}  
        on $M$ is a function $\eta$ such that $\eta(x)$ is an absolute $k$-tensor on $M_x$.
        Show that $\int_M \eta$ can be defined, even if $M$ is not orientable.
    }
    \problem{
        If $M_1\subset\F{R}^n$ is an $n$-dimensional manifold-with-boundary and $M_2\subset M_1-\partial M_1$ 
        is an $n$-dimensional manifold-with-boundary, and $M_1,M_2$ are compact, prove that 
        \begin{align*}
            \int_{\partial M_1}\omega=\int_{\partial M_2}\omega
        \end{align*}

        where $\omega$ is an $(n-1)$-form on $M_1$, and $\partial M_1$ and $\partial M_2$ 
        have the orientations induced by the usual orientations of $M_1$ and $M_2$. 
        \textit{Hint:} Find a manifold-with-boundary $M$ such that $\partial M=\partial M_1\cup \partial M_2$ 
        and such that the induced orientation on $\partial M$ agrees with that for
        $\partial M_1$ on $\partial M_1$ and is the negative of that for $\partial M_2$ on $\partial M_2$.
    }
\end{problems}