\clearpage
\section{THE CLASSICAL THEOREMS}
We have now prepared all the machinery necessary to state and prove the 
classical "Stokes' type" of theorems. We will indulge in a little bit of 
self-explanatory classical notation.

\begin{theorem}[Green's Theorem]
    Let $M \subset \F{R}^2$ be a compact two-dimensional manifold-with-boundary.
    Suppose that $\alpha,\beta:M\to\F{R}$ are differentiable. Then
    \begin{align*}
        \int_{\partial M}\alpha \dd x+\beta \dd y
        & = \int_{M}(\R{D}_1\beta-\R{D}_2\alpha)\;\dd x\wedge\dd y \\
        & = \iint_{M}\left(\frac{\partial\beta}{\partial x}-\frac{\partial\alpha}{\partial y}\right)\;\dd x\dd y
    \end{align*}
\end{theorem}

(Here $M$ is given the usual orientation, and $\partial M$ the induced
orientation, also known as the counterclockwise orientation.)

\begin{proof}
    This is a very special case of Theorem 5-5, 
    since $\dd(\alpha\dd x + \beta\dd y) =(\R{D}_1\beta-\R{D}_2\alpha)\dd x\wedge\dd y$.
\end{proof}

\begin{theorem}[Divergence Theorem]
    Let $M\subset\F{R}^3$ be a compact three-dimensional manifold-with-boundary and $n$ the
    unit outward normal on $\partial M$. Let $F$ be a differentiable vector field on $M$. Then
    \begin{align*}
        \int_M\div F\;\dd V = \int_{\partial M}\langle F,n\rangle\dd A
    \end{align*}

    This equation is also written in terms of three differentiable 
    functions $\alpha,\beta,\gamma:M\to\F{R}$:
    \begin{align*}
        \iiint_M\left(\frac{\partial\alpha}{\partial x}+\frac{\partial\beta}{\partial y}+\frac{\partial\gamma}{\partial z}\right)\;\dd V
        = \iint_{\partial M}(n^1\alpha+n^2\beta+n^3\gamma)\;\dd S    
    \end{align*}
\end{theorem}

\begin{proof}
    Define $\omega$ on $M$ by $\omega=F^1\dd y\wedge\dd z + F^2\dd z\wedge\dd x + F^3\dd x\wedge\dd y$.
    Then $\dd\omega = \div F\dd V$. According to Theorem 5-6, on $\partial M$ we have 
    \begin{align*}
        n^1\dd A & = \dd y\wedge \dd z\\
        n^2\dd A & = \dd z\wedge \dd x\\
        n^3\dd A & = \dd x\wedge \dd y
    \end{align*}

    Therefore on $\partial M$ we have 
    \begin{align*}
        \langle F,n\rangle\;\dd A 
        & = F^1n^1\dd A + F^2n^2\dd A + F^3n^3\dd A \\
        & = F^1\dd y\wedge \dd z + F^2\dd z\wedge \dd x + F^3\dd x\wedge \dd y \\
        & = \omega
    \end{align*}

    Thus, by Theorem 5-5 we have
    \begin{align*}
        \int_M\div F\;\dd V
        = \int_M\;\dd\omega
        = \int_{\partial M}\omega
        = \int_{\partial M}\langle F,n\rangle\;\dd A
    \end{align*}
\end{proof}

\begin{theorem}[Stoke's Theorem]
    Let $M\subset\F{R}^3$ be a compact oriented two-dimensional manifold-with-boundary and $n$ the
    unit outward normal on $M$ determined by the orientation of $M$. Let aM have the induced orientation.
    Let $T$ be the vector field on $\partial M$ with $\dd s(T) = 1$ and let $F$ be a differentiable 
    vector field in an open set containing $M$. Then
    \begin{align*}
        \int_{M}\langle (\nabla F,T), n\rangle\;\dd A = \int_{\partial M}\langle F,T\rangle\;\dd s
    \end{align*}

    This equation is sometimes written
    \begin{align*}
        & \int_{\partial M}\alpha \dd x+\beta\dd y+\gamma\dd z \\
        & = \iint_{M}\left[
            n^1\left(\frac{\partial\gamma}{\partial y}-\frac{\partial\beta}{\partial z}\right)
            + n^2\left(\frac{\partial\alpha}{\partial z}-\frac{\partial\gamma}{\partial x}\right)
            + n^3\left(\frac{\partial\beta}{\partial x}-\frac{\partial\alpha}{\partial y}\right)
            \right]\;\dd S
    \end{align*}
\end{theorem}

\begin{proof}
    Define $\omega$ on $M$ by $\omega=F^1\dd x + F^2\dd y + F^3\dd z$. Since $\nabla F\times F$ has 
    components $\R{D}_2F^3 - \R{D}_3F^2, \R{D}_3F^1-\R{D}_1F^3, \R{D}_1F^2-\R{D}_2F^1$, it follows,
    as in the proof of Theorem 5-8, that on $M$ we have 
    \begin{align*}
        \langle (\nabla F,T), n\rangle\;\dd A 
        & = (\R{D}_2F^3-\R{D}_3F^2)\;\dd y\wedge\dd z + (\R{D}_3F^1-\R{D}_1F^3)\;\dd z\wedge\dd x\\
        & \hspace*{11.4em} + (\R{D}_1F^2-\R{D}_2F^1)\;\dd x\wedge\dd y \\
        & = \dd\omega
    \end{align*}

    On the other hand, since $\dd s(T)=1$, on $\partial M$ we have 
    \begin{align*}
        T^1\;\dd s & = \dd x \\
        T^2\;\dd s & = \dd y \\
        T^3\;\dd s & = \dd z
    \end{align*}

    (These equations may be checked by applying both sides to $T_x$, 
    for $z\in\partial M$, since $T_x$ is a basis for $(\partial M)_x$.)

    Therefore on $\partial M$ we have
    \begin{align*}
        \langle F, T\rangle \;\dd s 
        & = F^1T^1\;\dd s + F^2T^2\;\dd s + F^3T^3\;\dd s \\
        & = F^1\dd x + F^2\dd y + F^3\dd z \\
        & = \omega
    \end{align*}

    Thus, by Theorem 5-5, we have 
    \begin{align*}
        \int_{M}\langle (\nabla F,T), n\rangle\;\dd A 
        = \int_{M}\;\dd\omega
        = \int_{\partial M}\omega
        = \int_{\partial M}\langle F,T\rangle\;\dd s
    \end{align*}
\end{proof}

Theorems 5-8 and 5-9 are the basis for the names $\div F$ and $\curl F$. If F(x) is the 
velocity vector of a fluid at $x$ (at some time) then $\int_{\partial M} \langle F,n\rangle\;\dd A$ 
is the amount of fluid ``diverging'' from $M$. Consequently the condition $\div F=0$ expresses
the fact that the fluid is incompressible. If M is a disc, then $\int_{\partial M} \langle F,T\rangle\;\dd S$
measures the amount that the fluid curls around the center of the disc. If this is zero for all discs, 
then $\nabla \times F=0$, and the fluid is called \textit{irrotational}.

These interpretations of $\div F$ and $\curl F$ are due to Maxwell \cite{maxwell1954electricity}.
Maxwell actually worked with the negative of $\div F$, which he accordingly 
called the \textit{convergence}. For $\nabla\times F$ Maxwell proposed ``with great diffidence'' 
the terminology rotation of $F$; this unfortunate term suggested the abbreviation
rot F which one occasionally still sees.

The classical theorems of this section are usually stated in
somewhat greater generality than they are here. For example, Green's Theorem 
is true for a square, and the Divergence Theorem is true for a cube. 
These two particular facts can be proved by approximating the 
square or cube by manifolds-with-boundary. A thorough generalization of the theorems of
this section requires the concept of manifolds-with-corners;
these are subsets of $\F{R}^n$ which are, up to diffeomorphism,
locally a portion of $\F{R}^k$ which is bounded by pieces of $(k-1)$-planes.
The ambitious reader will find it a challenging exercise to define manifolds-with-corners 
rigorously and to investigate how the results of this entire chapter may be generalized.

\newpage
\begin{problems}
    \problem{
        Generalize the divergence theorem to the case of an $n$-manifold with boundary in $\F{R}^n$.
    }
    \problem{
        Applying the generalized divergence theorem to the set $M=\{x\in\F{R}^n:|x|<a\}$ and 
        $F(x)=x_x$, find the volume of $S^{n-1} = \{x\in\F{R}^n:|x|=1\}$ in terms of the $n$-dimensional
        volume of $B_n=\{x\in\F{R}^n:|x|<1\}$. This volume is 
        \begin{align*}
            \frac{\pi^{\frac{n}{2}}}{(\frac{n}{2})!}
        \end{align*}
        if $n$ is even and 
        \begin{align*}
            \frac{2^{\frac{n+1}{2}}\cdot \pi^{\frac{n-1}{2}}}{1\times 3\times 5\times\cdots\times n}
        \end{align*}
        if $n$ is odd.
    }
    \problem{
        Define $F$ on $\F{R}^3$ by $F(x) = (0,0,cx^3)_x$ and let $M$ be a compact
        three-dimensional manifold-with-boundary with $M\subset\{x: x^3\le 0\}$.
        The vector field $F$ may be thought of as the downward pressure of a fluid 
        of density $c$ in $\{x: x^3\le 0\}$. Since a fluid exerts equal pressures in 
        all directions, we define the buoyant force on $M$, due to the fluid, 
        as $-\int_{\partial M}\langle F,n\rangle\;\dd A$. Prove the following theorem.
        \textit{Theorem (Archimedes)}. The buoyant force on $M$ is equal to the
        weight of the fluid displaced by $M$.
    }
\end{problems}
