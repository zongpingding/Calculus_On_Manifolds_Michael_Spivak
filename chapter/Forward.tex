% \chapter{Editors' Forewor}
% other way
\thispagestyle{empty}
\MakeLinkTarget*{editors-foreword}
\pdfbookmark{Editors' Foreword}{editors-foreword}
\topskip0pt
\vspace*{4em}
\begin{center}
  {\bfseries\sffamily\Huge Editors' Foreword}
\end{center}
\vspace*{5em}
  Mathematics has been expanding in all directions at a fabulous
rate during the past half century.New fields have emerged,
the diffusion into ether disciplines has proceeded apace, and
our knowledge of the classical areas has grown ever more profound. 
At the same time, one of the most striking trends in
modern mathematics is the constantly increasing interrelationship 
between its various branches. Thus the present day
students of mathematics are faced with an immense mountain
of material. In addition to the traditional areas of mathematics 
as presented in the traditional manner-and these
presentations do abound-there are the new and often enlightening 
ways of looking at these traditional areas, and also
the vast new areas teeming with potentialities. Much of this
new material is scattered indigestibly throughout the research
journals, and frequently coherently organized only in the
minds or unpublished notes of the working mathematicians.
And students desperately need to learn more and more of this
material.

    This series of brief topical booklets has been conceived as a
possible means to tackle and hopefully to alleviate some of 
these pedagogical problems. They are being written by active
research mathematicians, who can look at the latest developments, 
who can use these developments to clarify and condense the required material, 
who know what ideas to under score and what techniques to stress. We hope that they will
also serve to present to the able undergraduate an introduction
to contemporary research and problems in mathematics, and
that they will be sufficiently informal that the personal tastes
and attitudes of the leaders in modern mathematics will shine
through clearly to the readers. 

The area of differential geometry is one in which recent developments 
have effected great changes. That part of differential geometry centered 
about Stokes' Theorem, sometimes called the fundamental theorem of multivariate 
calculus,is traditionally taught in advanced calculus courses (second or
third year) and is essential in engineering and physics as well
as in several current and important branches of mathematics.
However, the teaching of this material has been relatively
little affected by these modern developments; so the mathematicians 
must relearn the material in graduate school, and
other scientists are frequently altogether deprived of it. Dr.
Spivak's book should be a help to those who wish to see
Stoke's Theorem as the modern working mathematician sees
it. A student with a good course in calculus and linear 
algebra behind him should find this book quite accessible.

\vspace*{1em}
\hspace*{\fill}\textbf{\itshape Robert Gunning}\par 
\hspace*{\fill}\textbf{\itshape Hugo Rossi}\par 
\noindent {\itshape Princeton, New Jersey}\par 
\noindent {\itshape Waltham, Massachusetts}\par 
\noindent {\itshape August 1965}
\newpage
