\section[\textsc{norm and inner product}]{NORM AND INNER PRODUCT}\index{Inner product}
\textbf{Euclidean} $n$-\textbf{space} $\F{R}^n$ is defined as the set of all
$n$-tuples $(x^1,\cdots,x^n)$ of real numbers $x^i$ (a "1-tuple of numbers" is 
just a number and $\F{R}^1 = \F{R}$, the set of all real numbers). 
An element of $\F{R}^n$ is often called a point in $\F{R}^n$, and $\F{R}^1$, $\F{R}^2$, $\F{R}^3$
are often called the line\index{Line}, the plane\index{Plane}, and space\index{Space}\indexseealso{Space}{Dual space, Euclidean
space, Half-space, Tangent space}, respectively. 
If $x$ denotes an element of $\F{R}^n$, then $x$ is an $n$-tuple of numbers, the
$i$th one of which is denoted $x^i$; thus we can write 
\begin{align*}
  x = (x^1, x^2, \cdots, x^n).
\end{align*}

A Point\index{Point} in $\F{R}^n$ is frequently also called a vector\index{Vector} in $\F{R}^n$,
because $\F{R}^n$, with $x+y = (x^1+y^1, \cdots, x^n+y^n)$ and $ax=(ax^1, \cdots, ax^n)$,
as operations, is a vector space(over the real numbers, of dimension $n$).
In this vector space there is the notion of the \textbf{length}\index{Length $=$ norm} of a vector $x$, usually 
called the \Index{Norm}\index{Absolute value} $|x|$ of $x$ and 
defined by $|x| = \sqrt{(x^1)^2 +\cdots + (x^n)^2}$. If $n=1$, then $|x|$ is the usual absolute 
value of $x$. The relation between the norm and the vector space structure of $\F{R}^n$ is very 
important.

\begin{theorem}
  If $x, y \in \F{R}^n$ and $a\in \F{R}$, then 
  \begin{enumerate}[label={\upshape(\arabic*)}]
      \item $|x| \ge 0$, and $|x| = 0$ if and only if $x=0$ 
      \item $\sum_{i=1}^{n }{x^iy^i} \le |x|\cdot |y|$; equality holds if and only if 
        $x$ and $y$ are linearly dependent.
      \item $|x+y| \le |x| + |y|$.
      \item $|ax| = |a|\cdot |x|$.
  \end{enumerate}
\end{theorem}

\begin{proof}
    \begin{enumerate}[label={\upshape(\arabic*)}]
        \item is left to the reader.
        \item If $x$ and $y$ are linearly dependent, equality clearly holds.
            If not, then $\lambda y - x\neq 0$  for all $\lambda\in \F{R}$ , so
            \begin{align*}\begin{aligned}
                0 < |\lambda y - x|^2 
                & = \sum_{i=1 }^{n }{(\lambda y^i - x^i)^2} \\
                & = \lambda^2 \sum_{i=1 }^{n }{(y^i)^2 - 2\lambda \sum_{i=1 }^{n }{x^iy^i}} + \sum_{i=1 }^{n }{(x^i)^2}.
            \end{aligned}\end{align*}
            Therefore the right side is a quadratic equation in $\lambda$ with no
            real solution, and its discriminant must be negative. Thus
            \begin{align*}
                4\Bigl(\sum_{i=1 }^{n }{x^iy^i}\Bigr)^2 - 4 \sum_{i=1 }^{n }{(x^i)^2}\cdot \sum_{i=1 }^{n }{(y^i)^2} < 0.
            \end{align*}
        \item \quad \vspace*{-2.75em}
            \begin{align*}\begin{aligned}
                \left|x+y\right|^2 
                & = \sum_{i=1 }^{n }{(x^i + y^i)^2} \\
                & = \sum_{i=1 }^{n }{(x^i)^2} + \sum_{i=1 }^{n }{(y^i)^2} +  2\sum_{i=1 }^{n }{x^iy^i} \\
                & \le |x|^2 + |y|^2 + 2|x|\cdot |y| \qquad \text{by (2)}\\
                & = \left(|x| + |y|\right)^2.
            \end{aligned}\end{align*}
        \item %\quad \vspace*{-2.75em}
            \begin{align*}\begin{aligned}
                |ax| 
                = \sqrt{\sum_{i=1 }^{n }{(ax^i)^2}} 
                = \sqrt{a^2\cdot \sum_{i=1}^{n }{(x^i)^2}}
                = |a|\cdot |x|.
            \end{aligned}\end{align*}
    \end{enumerate}
\end{proof}

The quantity $\sum_{i=1 }^{n }{x^iy^i}$ which appears in (2) is called the 
\textbf{inner product} of $x$ and $y$ and denoted $\langle x, y\rangle$.
The most important properties of the inner product are the following.


\begin{theorem}
    If $x, x_1, x_2$ and $y, y_1, y_2$ are vectors in $\F{R}^n$ and 
    $a\in \F{R}$, then 
    \begin{enumerate}[label={\upshape(\arabic*)}]
        \item \text{symmetry:}\index{Symmetric}
            \begin{align*}
                \langle x, y\rangle = \langle y, x\rangle
            \end{align*}
        \item \text{bilinearity:}\index{Bilinear function}
            \begin{align*}\begin{aligned}
                & \langle ax, y\rangle = \langle x, ay\rangle = a\langle x, y\rangle\\
                & \langle x_1+x_2, y\rangle = \langle x_1, y\rangle + \langle x_2, y\rangle \\
                & \langle x, y_1+y_2\rangle = \langle x, y_1\rangle + \langle x, y_2\rangle
            \end{aligned}\end{align*}
        \item \text{positive definiteness:}\index{Positive definiteness}
            \begin{align*}
                \langle x, y\rangle\ge 0
            \end{align*}    
            and $\langle x, x\rangle = 0$ if and only if $x=0$
        \item $|x| = \sqrt{\langle x, x\rangle}$
        \item \text{polarization identity:}\index{Polarization identity}
            \begin{align*}
              \langle x, x\rangle = \frac{|x+y|^2 - |x-y|^2}{4}
            \end{align*}
    \end{enumerate} 
\end{theorem}

\begin{proof}
    \begin{enumerate}[label={\upshape(\arabic*)}]
        \item $\langle x,y\rangle=\Sigma_{i=1}^{n}x^{i}y^{i}=\Sigma_{i=1}^{n}y^{i}x^{i}=\langle y,x\rangle $
        \item By (1) it suffices to prove
            \begin{align*}\begin{aligned}
                & \langle ax,y\rangle = {a\langle x,y\rangle,}\\
                & \langle x_{1}+x_{2}, y\rangle = \langle x_{1},y\rangle+\langle x_{2},y\rangle
            \end{aligned}\end{align*}
            These follow from the equations
            \begin{align*}\begin{aligned}
                \langle ax,y\rangle
                    & =\sum_{i=1}^{n}(ax^{i})y^{i}
                        = a\sum_{i=1}^{n}x^{i}y^{i}=a\langle x,y\rangle,\\
                \langle x_{1}+x_{2},y\rangle
                    & =\sum_{i=1}^{n}(x_{1}{}^{i}+x_{2}{}^{i})y^{i}
                        = \sum_{i=1}^{n}x_{1}{}^{i}y^{i}+\sum_{i=1}^{n}x_{2}{}^{i}y^{i} \\
                    & = \langle x_{1},y\rangle+\langle x_{2},y\rangle.
            \end{aligned}\end{align*}
        \item and (4) are left to the reader.
        \item \quad \vspace*{-2em}
            \begin{align*}\begin{aligned}
                & \frac{|x+y|^2-|x-y|^2}4 \\
                & = \frac14[\langle x+y,x+y\rangle-\langle x-y,x-y\rangle]\quad\mathrm{by~}(4) \\
                & = \frac14[\langle x,x\rangle+2\langle x,y\rangle+\langle y,y\rangle-(\langle x,x\rangle-2\langle x,y\rangle+\langle y,y\rangle)] \\
                & = \langle x,y\rangle.
            \end{aligned}\end{align*}
    \end{enumerate}
\end{proof}

We conclude this section with some important remarks
about notation\index{Notation}. The vector $(0, . . . ,0)$ will usually be
denoted simply 0. The \textbf{usual basis}\index{Basis, usual for $\F{R}^n$} of $\F{R}^n$ is $e_1, \cdots, e_n$, 
where $e_i = (0, \cdots, 1, \cdots, 0)$, with $1$ in the $i$th place.
if $T:\F{R}^n \to \F{R}^m$ is a linear transformation, the matrix\index{Matrix} of 
$T$ with respect to the usual bases of $\F{R}^n$ and $\F{R}^m$ is the 
$m\times n$ matrix $A = (a_{ij})$, where $T(e_i) = \sum_{j=1 }^{m }{a_{ji }e_j}$ 
-- the coefficients of $T(e_i)$ appear in the $i$th column of the matrix.
If $S:\F{R}^n \to \F{R}^m$ has the $p\times m$ matrix $B$, then $S\circ T$ has 
the $p\times n$ matrix $BA$[here $S\circ T(x) = S(T(x))$; most books on linear 
algebra denote $S\circ T$ simply $ST$].
To find $T(x)$ one computes the $m\times 1$ matrix
\begin{align*}
    \begin{pmatrix}
        y^1\\.\\.\\.\\y^m
    \end{pmatrix}
    =
    \begin{pmatrix}
        a_{11} & \cdot & \cdot & \cdot & a_{1n} \\
        \cdot  &       &       &       & \cdot \\
        \cdot  &       &       &       & \cdot \\
        \cdot  &       &       &       & \cdot \\
        a_{m1} & \cdot & \cdot & \cdot & a_{mn}
    \end{pmatrix}
    \cdot
    \begin{pmatrix}
        x^1\\.\\.\\.\\x^n
    \end{pmatrix}
\end{align*}  

then $T(x) = (y^1, \cdots, y^m)$. One notational convention greatly 
simplifies many formulas: if $x\in \F{R}^n$ and $y\in \F{R}^m$, then 
$(x, y)$ denotes 
\begin{align*}
  (x^1,...,x^n,y^1,...,y^m)\in\mathbb{R}^{n+m}
\end{align*}


\begin{problems}
  \problem[*]{Prove that $|x| \le \sum_{i=1 }^{n }{|x^i|}$.}
  \problem{When does equality hold in Theorem 1-1(3)? \textit{Hint:} 
    Re-examine the proof; the answer is not "when $x$ and $y$ are linearly depend-ent."}
  \problem{Prove that $|x-y| \le |x| + |y|$. When does the equality holds?}
  \problem{Prove that $\left||x| - |y|\right| \le |x-y|$.}
  \problem{The quantity $|y-x|$ is called the \textbf{distance}\index{Distance} between $x$ and $y$.Prove and 
    interpret geometrically the ``triangle inequality''\index{Triangle inequality}: $|z-x| \le |z-y| + |y-x|$.}
  \problem{Let $f$ and $g$ be integrable on $[a, b]$.
    \begin{enumerate}[label={\upshape(\alph*)}]
      \item Prove that 
        \begin{align*}
          \bigg|\int_{a }^{b }{f\cdot g }\bigg| \le \Big(\int_{a }^{b }{f^2}\Big)^{\frac{1 }{2}} \cdot \Big(\int_{a }^{b }{g^2}\Big)^{\frac{1 }{2}}
        \end{align*} 
        \textit{Hint:} Consider separately the cases $0=\int_{a }^{b }{\left(f-\lambda g\right)^2}$ for some $\lambda \in \F{R}$
        and $0< \int_{a }^{b }{\left(f-\lambda g\right)^2}$ for all $\lambda \in \F{R}$
      \item If equality holds, must $f=\lambda g$ for some $\lambda \in \F{R}$? What if $f$ and $g$ are continuous?
      \item Show that Theorem 1-1(2) is a special case of (a).  
    \end{enumerate}}
  \problem{A linear transformation $T:\F{R}^n \to \F{R}^m$ is \textbf{norm preserving}\index{Norm preserving} if $|T(x)| = |x|$, and 
    \textbf{inner product preserving}\index{Inner product!preserving} if $\langle Tx, Ty\rangle = \langle x, y\rangle$.
    \begin{enumerate}[label={\upshape(\alph*)}]
        \item Prove that $T$ is norm preserving if and only if $T$ is inner-product preserving.
        \item Prove that such a linear transformation $T$ is $1-1$ and $T^{-1}$ is of the same sort
    \end{enumerate}}
  \problem{If $x,y\in\mathbb{R}^n$ are non-zero, the \textbf{angle}\index{Angle} between $x$ and $y, $ denoted $\angle(x,y)$, is defined as 
    $\arccos(\langle x,y\rangle/|x|\cdot|y|)$, which makes sense by Theorem 1-1(2). The linear transformation $T$ is 
    \textbf{angle preserving}\index{Angle!preserving} if $T$ is 1-1,and for $x, y\neq0$ we have $\angle(Tx,Ty)=\angle(x,y).$
    \begin{enumerate}[label={\upshape(\alph*)}]
        \item Prove that if $T$ is norm preserving, then $T$ is angle preserving.
        \item If there is a basis $x_1,\dots,x_n$ of $\F{R}^n$ and numbers $\lambda_1,\dots,\lambda_n$ 
          such that $Tx_i=\lambda_ix_i$, prove that $T$ is angle preserving if and only if all $|\lambda_i|$
          are equal.
        \item What are all angle preserving $T:\F{R}^n \to \F{R}^m$ ?
    \end{enumerate}}
  \problem{If $0\le \theta < \pi$, let $T:\F{R}^2\to \F{R}^2$ have the matrix $
      \begin{pmatrix}\cos\theta & \sin\theta \\ -\sin\theta & \cos\theta\end{pmatrix}$
      Show that $T$ is a angle preserving and if $x\neq 0$, then $\angle(x, Tx) = \theta$.} 
  \problem[*]{If $T:\F{R}^n\to \F{R}^m$ is a linear transformation, show that there is a number $M$ such that 
      $\left|T(h)\right| \le M|h|$ for $h\in \F{R}^m$. \textit{Hint:} Estimate $\left|T(h)\right|$ in trems of $|h|$ 
      and the entires in the matrix of $T$.}
  \problem{If $x, y\in \F{R}^n$ and $z, w\in \F{R}^m$, show that $\langle (x, z), (y, w)\rangle = \langle x, y\rangle + \langle z, e\rangle$
      and $|(x, z)| = \sqrt{|x|^2 + |z|^2}$. Note that $(x, z)$ and $(y, w)$ denote points in $\F{R}^{m+n}$}
  \problem[*]{Let $\left(\F{R}^n\right)^*$ denote the dual space\index{Dual space} of the vector $\F{R}^n$. If $x\in \F{R}^n$,
      define $\varphi_x\in \left(\F{R}^n\right)^*$ by $\varphi_x(y) = \langle x, y\rangle$. Define $T:\F{R}^n\to \left(\F{R}^n \right)^*$
      by $T(x) = \varphi_x$. Show that $T$ is a 1-1 linear transformation and conclude that every $\varphi \in \left(\F{R}^n \right)^*$
      is $\varphi_x$ for a unique $x\in \F{R}^n$}
  \problem[*]{If $x, y\in \F{R}^n$, then $x$ and $y$ are called \textbf{perpendicular}\index{Perpendicular} (or \textbf{orthogonal}\index{Orthogonal}) 
      if $\langle x, y\rangle = 0$. If $x$ and $y$ are perpendicular, prove that $|x+y|^2 = |x|^2 + |y|^2$}
\end{problems}