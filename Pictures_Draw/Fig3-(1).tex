% 生成最终的图象时把第一个文档类取消注释即可
\documentclass[10pt,varwidth]{standalone}
% \documentclass[12pt]{article}
% 1.必须添加varwidth选项,不然就会报错
\PassOptionsToPackage{quiet}{fontspec}
\usepackage{ctex}
\usepackage{xfp}
% 必须要保证绘图的纸张足够的大
\usepackage[a3paper, left=2.5cm, right=2.5cm, top=2.5cm, bottom=2.5cm]{geometry}
\usepackage{xifthen}
\usepackage{xfp}
\usepackage{xcolor}
\usepackage{pgfplots}
\usepackage{pgfplotstable}
\pgfplotsset{compat=1.16}
% 2.引用的tikz库
\usetikzlibrary {matrix, chains, trees, decorations}
\usetikzlibrary {arrows.meta, automata,positioning}
\usetikzlibrary {decorations.pathmorphing, calc}
\usetikzlibrary {calligraphy}
\usetikzlibrary {backgrounds, mindmap,shadows}
\usetikzlibrary {patterns, quotes, 3d, shadows}
\usetikzlibrary {graphs, fadings, scopes}
\usetikzlibrary {arrows, shapes.geometric}
\usepgflibrary {shadings}

% \tikzset{
%     >={Latex[length=6mm, width=2mm]}
% }

\begin{document}
\quad 

\begin{tikzpicture}[>=Latex]
\foreach \line in {1, 2, 3, 4, 5} {
    \foreach \col in {1, 2, 3, 4}{
        \node at (\fpeval{\line*2}, \fpeval{\col*2}) {\fpeval{\line-1}/\fpeval{5-\col}};
        \ifnum\fpeval{\col>\line-1?1:0}=1
            \ifnum\fpeval{\col=4?1:0}=1
                \draw[->] (\fpeval{\line*2}, \fpeval{\col*2})++(.35,.35) -- ($(\line*2+2, \col*2+2)+(-.35, -.35)$);
            \else
                \draw (\fpeval{\line*2}, \fpeval{\col*2})++(.35,.35) -- ($(\line*2+2, \col*2+2)+(-.35, -.35)$);
            \fi
            \draw (\fpeval{\line*2-2}, \fpeval{\col*2-2})++(.35,.35) -- ($(\line*2, \col*2)+(-.35, -.35)$);
        \fi
    }
}
\end{tikzpicture}
\end{document}