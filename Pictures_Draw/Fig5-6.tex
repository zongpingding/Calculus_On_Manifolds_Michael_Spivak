% 生成最终的图象时把第一个文档类取消注释即可
\documentclass[10pt,varwidth]{standalone}
% \documentclass[12pt]{article}
% 1.必须添加varwidth选项,不然就会报错
\PassOptionsToPackage{quiet}{fontspec}
\usepackage{ctex}
% 必须要保证绘图的纸张足够的大
\usepackage[a3paper, left=2.5cm, right=2.5cm, top=2.5cm, bottom=2.5cm]{geometry}
\usepackage{xifthen}
\usepackage{xfp}
\usepackage{xcolor}
\usepackage{pgfplots}
\usepackage{pgfplotstable}
\pgfplotsset{compat=1.16}
% 2.引用的tikz库
\usetikzlibrary {matrix, chains, trees, decorations}
\usetikzlibrary {arrows.meta, automata,positioning}
\usetikzlibrary {decorations.pathmorphing, calc}
\usetikzlibrary {calligraphy}
\usetikzlibrary {backgrounds, mindmap,shadows}
\usetikzlibrary {patterns, quotes, 3d, shadows}
\usetikzlibrary {graphs, fadings, scopes}
\usetikzlibrary {arrows, shapes.geometric}
\usepgflibrary {shadings}
\usetikzlibrary{angles}

\tikzset{
    >={Latex}
}



\begin{document}
\;\begin{tikzpicture}[rotate=-45]
    % draw axis
    % \draw[->] (-5,0) -- (5,0) node[right] {$x$};
    % \draw[->] (0,-5) -- (0,5) node[above] {$y$};
    % above left
    \begin{scope}
        \begin{scope}[xshift=-1cm, yshift=0]
            \draw[->] (4, 2) coordinate (O)  -- (2, 2) coordinate (f1);
            \draw[->] (4, 2) -- (4, 4) coordinate (f2);
            \pic [draw, ->, angle radius = 8mm, angle eccentricity=0, ""] {angle = f2--O--f1};    
        \end{scope}
        \begin{scope}[xshift=-5cm, yshift=0cm]
            \draw[->] (4, 2) coordinate (O)  -- (2, 2) coordinate (f1);
            \draw[->] (4, 2) -- (4, 4) coordinate (f2);
            \pic [draw, ->, angle radius = 8mm, angle eccentricity=0, ""] {angle = f2--O--f1};    
        \end{scope}
        \begin{scope}[xshift=-5cm, yshift=-5cm]
            \draw[->] (4, 2) coordinate (O)  -- (2, 2) coordinate (f1);
            \draw[->] (4, 2) -- (4, 4) coordinate (f2);
            \pic [draw, ->, angle radius = 8mm, angle eccentricity=0, ""] {angle = f2--O--f1};    
        \end{scope}
        \begin{scope}[xshift=-1cm, yshift=-5cm]
            \draw[->] (4, 2) coordinate (O)  -- (2, 2) coordinate (f1);
            \draw[->] (4, 2) -- (4, 4) coordinate (f2);
            \pic [draw, ->, angle radius = 8mm, angle eccentricity=0, ""] {angle = f2--O--f1};    
        \end{scope}
        \begin{scope}[xshift=.5cm, yshift=-6cm]
            \draw[->] (4, 2) coordinate (O)  -- (2, 2) coordinate (f1);
            \draw[->] (4, 2) -- (4, 4) coordinate (f2);
            \pic [draw, ->, angle radius = 8mm, angle eccentricity=0, ""] {angle = f2--O--f1};    
        \end{scope}
        \draw (5, 4.5) to[out=170, in=10] (-4.5, 4.5);
        \draw (-4.5, 4.5) to[out=-85, in=100] (-4.5, -6);
        \draw (-4.5, -6) to[out=3, in=-185] (5.5, -5.5);
        \draw (5.5, -5.5) to[out=100, in=-85] (5, 4.5);
    \end{scope}
    % above right
    \begin{scope}[xshift=13cm]
        \begin{scope}[xshift=-1cm, yshift=0]
            \draw[->] (3, 2) coordinate (O)  -- (5, 2) coordinate (f1);
            \draw[->] (3, 2) -- (3, 4) coordinate (f2);
            \pic [draw, ->, angle radius = 8mm, angle eccentricity=0, ""] {angle = f1--O--f2};    
        \end{scope}
        \begin{scope}[xshift=-5.5cm, yshift=.5cm]
            \draw[->] (4, 2) coordinate (O)  -- (2, 2) coordinate (f1);
            \draw[->] (4, 2) -- (4, 4) coordinate (f2);
            \pic [draw, ->, angle radius = 8mm, angle eccentricity=0, ""] {angle = f2--O--f1};    
        \end{scope}
        \begin{scope}[xshift=-4.5cm, yshift=-.5cm]
            \draw[->] (4, 2) coordinate (O)  -- (2, 2) coordinate (f1);
            \draw[->] (4, 2) -- (4, 4) coordinate (f2);
            \pic [draw, ->, angle radius = 8mm, angle eccentricity=0, ""] {angle = f2--O--f1};    
        \end{scope}
        \begin{scope}[xshift=-5cm, yshift=-5cm]
            \draw[->] (4, 2) coordinate (O)  -- (2, 2) coordinate (f1);
            \draw[->] (4, 2) -- (4, 4) coordinate (f2);
            \pic [draw, ->, angle radius = 8mm, angle eccentricity=0, ""] {angle = f2--O--f1};    
        \end{scope}
        \begin{scope}[xshift=-0.5cm, yshift=-4.5cm]
            \draw[->] (3, 2) coordinate (O)  -- (5, 2) coordinate (f1);
            \draw[->] (3, 2) -- (3, 4) coordinate (f2);
            \pic [draw, <-, angle radius = 8mm, angle eccentricity=0, ""] {angle = f1--O--f2};    
        \end{scope}
        \begin{scope}[xshift=4cm, yshift=-6.5cm, rotate=90]
            \draw[->] (4, 2) coordinate (O)  -- (2, 2) coordinate (f1);
            \draw[->] (4, 2) -- (4, 4) coordinate (f2);
            \pic [draw, ->, angle radius = 8mm, angle eccentricity=0, ""] {angle = f2--O--f1};    
        \end{scope}
        \draw (5, 4.5) to[out=170, in=10] (-4.5, 4.5);
        \draw (-4.5, 4.5) to[out=-85, in=100] (-4.5, -6);
        \draw (-4.5, -6) to[out=3, in=-185] (5.5, -5.5);
        \draw (5.5, -5.5) to[out=100, in=-85] (5, 4.5);
    \end{scope}
    \node at (6, -6) {(a)};
    \node at (19, -6) {(b)};
\end{tikzpicture}
\end{document}