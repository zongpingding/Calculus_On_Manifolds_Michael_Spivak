% 生成最终的图象时把第一个文档类取消注释即可
\documentclass[10pt,varwidth]{standalone}
% \documentclass[12pt]{article}
% 1.必须添加varwidth选项,不然就会报错
\PassOptionsToPackage{quiet}{fontspec}
\usepackage{ctex}
% 必须要保证绘图的纸张足够的大
\usepackage[a4paper, left=2.5cm, right=2.5cm, top=2.5cm, bottom=2.5cm]{geometry}
\usepackage{xifthen}
\usepackage{xfp}
\usepackage{xcolor}
\usepackage{pgfplots}
\usepackage{pgfplotstable}
\pgfplotsset{compat=1.16}
% 2.引用的tikz库
\usetikzlibrary {matrix, chains, trees, decorations}
\usetikzlibrary {arrows.meta, automata,positioning}
\usetikzlibrary {decorations.pathmorphing, calc}
\usetikzlibrary {calligraphy}
\usetikzlibrary {backgrounds, mindmap,shadows}
\usetikzlibrary {patterns, quotes, 3d, shadows}
\usetikzlibrary {graphs, fadings, scopes}
\usetikzlibrary {arrows, shapes.geometric}
\usepgflibrary {shadings}

\tikzset{
    >={Latex}
}



\begin{document}
\;

\begin{tikzpicture}
    \begin{scope}
        \filldraw (1, 0) circle (1pt);
        \filldraw (4, 0) circle (1pt);
        \draw (0, 0) -- (5, 0) node[midway, below=2em] {$(a)$};
        \draw[thick] (1, 0)node[below] {$I^1_{(1,0)}$} -- (4, 0)node[below] {$I^1_{(1,1)}$};
    \end{scope}
    \begin{scope}[xshift=8cm]
        \draw[->] (0, 0) -- (0, 5);
        \draw[->] (0, 0) -- (5, 0);
        \draw[->] (3, 0) -- (3, 3) node[midway, right] {$I^2_{(1,1)}$};
        \draw[->] (0, 0) -- (3, 0) node[midway, below=2em] {$(b)$}; 
        \draw[->] (0, 0) -- (3, 0) node[midway, below] {$I^2_{(2,0)}$}; 
        \draw[->] (3, 3) -- (0, 3) node[midway, above] {$I^2_{(2,1)}$};
        \draw[->] (0, 3) -- (0, 0) node[midway, left] {$I^2_{(1,0)}$};
    \end{scope}
\end{tikzpicture}
\end{document}