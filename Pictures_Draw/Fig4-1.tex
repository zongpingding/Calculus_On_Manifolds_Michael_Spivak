% 生成最终的图象时把第一个文档类取消注释即可
\documentclass[10pt,varwidth]{standalone}
% \documentclass[12pt]{article}
% 1.必须添加varwidth选项,不然就会报错
\PassOptionsToPackage{quiet}{fontspec}
\usepackage{ctex}
% 必须要保证绘图的纸张足够的大
\usepackage[a4paper, left=2.5cm, right=2.5cm, top=2.5cm, bottom=2.5cm]{geometry}
\usepackage{xifthen}
\usepackage{xfp}
\usepackage{xcolor}
\usepackage{pgfplots}
\usepackage{pgfplotstable}
\pgfplotsset{compat=1.16}
% 2.引用的tikz库
\usetikzlibrary {matrix, chains, trees, decorations}
\usetikzlibrary {arrows.meta, automata,positioning}
\usetikzlibrary {decorations.pathmorphing, calc}
\usetikzlibrary {calligraphy}
\usetikzlibrary {backgrounds, mindmap,shadows}
\usetikzlibrary {patterns, quotes, 3d, shadows}
\usetikzlibrary {graphs, fadings, scopes}
\usetikzlibrary {arrows, shapes.geometric}
\usepgflibrary {shadings}

\tikzset{
    >={Latex}
}



\begin{document}
\begin{tikzpicture}
    % coordinate
    \draw[->] (-5, 0) -- (5, 0);
    \draw[->] (0, -5) -- (0, 5);
    % vectors
    \begin{scope}
        \draw[->] (0, 0) -- (2.5, 2.5)node[above right] {$v$};
    \end{scope}
    \begin{scope}[xshift=2.6cm, yshift=1cm]
        \draw[->] (0, 0) -- (2.5, 2.5)node[midway, below]{$v_p$};
    \end{scope}
    % annotation
    \filldraw (0, 0) circle (1pt);
    \filldraw (2.6, 1) circle (1pt)node[below left]{$p$};
    \node[above right] at (5.1, 3.5) {$p + v$};
\end{tikzpicture}
\end{document}